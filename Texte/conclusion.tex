\begin{conclusion}
En guise de conclusion, ce projet montre qu'il est possible de contrôler un bras robotique à l'aide d'une souris 3D.
% Il est agréable de constater l'impact du logiciel libre, sans logiciels libres, ce projet n'aurait pas été réalisé dans d'aussi bref délais. 
Il aurait bien entendu été intéressant de réussir à établir la communication entre le logiciel et le matériel, mais la preuve de concept sur simulateur est suffisante en elle-même afin de déclarer ce projet un succès. 

Ce projet permis d'appliquer les notions apprises dans les cours GPA546 (Robots industriels) et GPA434 (Ingénierie des systèmes orientés objet). Il est important de mentionner que la matière du défunt cours GPA435 (Systèmes d'exploitation et programmation de systèmes) aurait fort probablement aidé à régler certains problèmes rencontrés au cours de ce projet puisque l'entièreté du logiciel développé ici repose sur Linux.

Dans le cadre d'une mission de recherche et sauvetage, l'algorithme développé par ce projet permettra à l'opérateur de sauver de précieuses minutes sur les manipulations. Malgré qu'elle n'ait pas été mesurée, la réduction de la charge mentale est flagrante par rapport au contrôle joint par joint. L'algorithme habilitera aussi l'opérateur à effectuer des manipulations préalablement impossible. Ce domaine étant lié au militaire, l'exemple le plus fréquent de manipulation sensible est la manipulation d'une bombe tuyau%\footnote{\href{https://en.wikipedia.org/wiki/Pipe_bomb}{wikipedia.org/wiki/Pipe\_bomb}}
. L'algorithme développé permettant de tirer dans un axe précis, ceci rend l'opération plus sécuritaire car il réduit le risque d'explosion soudaine et non contrôlée. 

En terminant, il serait éventuellement intéressant de comparer l'efficacité de l'algorithme développé dans le cadre de ce projet avec l'algorithme jog\_arm\footnote{\href{https://ros-planning.github.io/moveit_tutorials/doc/arm_jogging/arm_jogging_tutorial.html}{ros-planning.github.io/moveit\_tutorials/doc/arm\_jogging/arm\_jogging\_tutorial.html}} qui fut ajouté à la documentation officielle de MoveIt récemment.
\end{conclusion}