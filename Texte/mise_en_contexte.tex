Les services d'urgence sont appelés à travailler dans des conditions extrêmes. Dans le but de réduire leur exposition à certain types de danger tel que les radiations, sols instables, températures élevées, explosifs, ceux-ci utilisent des robots. L'utilisation de robots permet aux répondants de rester à une distance sécuritaire du danger. Toutefois, le temps de réponse et l'efficacité d'une intervention dépendent principalement des deux variables suivantes soit l'aisance de contrôle et la capacité pour l'opérateur à visualiser l'environnement qui entoure le robot.\footnote{\cite{massey_improved_2009}}

Par conséquent, le contrôle du robot devra être conçu de manière à être intuitif. Ceci dans le but de permettre à un opérateur n'ayant aucune expérience préalable, de maîtriser le bras robotique très rapidement. Poursuivant l'idée explorée par la marine américaine pour le contrôle du périscope de leurs sous-marins\footcite{brock_vergakis_navys_2017}, nous allons utiliser du matériel pré-existant et facile d'utilisation. Le but d'utiliser un composant en vente libre est qu'en raison des chaînes d'approvisionnement globales, il est possible de se procurer et de remplacer rapidement le matériel peu importe l'emplacement géographique où se déroule la mission.

Il n'y aura pas d'étude visant à qualifier la charge cognitive résultante de ce projet. Ceci parce que selon nos recherches, il n'existe pas à l'heure actuelle de méthode permettant de mesurer avec précision la charge mentale imposée à un opérateur. Certaines méthodes d'analyses existent tel que le NASA Task Load Index\footcite{hart_nasa_1986} mais celle-ci n'est pas en mesure de réduire suffisamment le biais du répondant. Une autre méthode qui semble plus probante est la mesure par oculométrie\footcite{st-onges_equipes_robots_2020} mais nous ne l'aborderons pas dans ce rapport étant donné que ce n'est pas le sujet de ce projet.

% \subsection{Requis de conception}
% \begin{enumerate}[align=left]
%     \item [\emph{Autonomous Flippers}] La compétition offre un multiplicateur
%     supplémentaire si les "flippers" se contrôlent sans une action de l'opérateur. Ceux-ci doivent être contrôlée selon des instructions strictes énoncés dans le cahier des règles de la compétition.\footcite{}
%     \item [\emph{Le robot doit pouvoir reculer}] Pour certains parcours de la compétition (MAN), le robot doit effectuer l'aller-retour sans pivoter sur lui-même.
% \end{enumerate}

